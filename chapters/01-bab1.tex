\chapter{PENDAHULUAN}

\section{Latar Belakang}

PT Lare Osing Ndo merupakan penyedia layanan internet lokal di Banyuwangi yang menggunakan infrastruktur \textit{switch} Layer 2 dengan teknologi VLAN. Dalam operasionalnya saat ini, perusahaan menghadapi tantangan serius dalam menjaga kestabilan layanan akibat kompleksitas manajemen jaringan yang terus meningkat. Seiring dengan tren pertumbuhan jumlah pengguna dan volume trafik internet, ISP regional seperti PT Lare Osing Ndo dituntut untuk tetap memenuhi \textit{Service Level Agreement} (SLA) yang telah ditetapkan. Dalam konteks ini, pendekatan berbasis kecerdasan buatan terdistribusi \cite{alhachem2025} mulai diperlukan. Lebih jauh lagi, untuk mewujudkan era \textit{zero downtime}, penggunaan \textit{machine learning} menjadi sangat penting guna memprediksi potensi kegagalan jaringan sebelum berkembang menjadi gangguan yang lebih parah \cite{basikolo2023}.

Salah satu permasalahan krusial yang dihadapi adalah pengelolaan trafik yang semakin kompleks, terutama setelah penambahan \textit{link} redundansi dalam jumlah besar. Sampai saat ini, penentuan jalur masih mengandalkan protokol konvensional atau bahkan keputusan manual dari administrator. Pendekatan seperti ini kurang efektif mengingat \textit{Traffic Engineering} (TE) modern membutuhkan kemampuan untuk menangani pola trafik yang dinamis dan beragam \cite{marouani2024}, sesuatu yang sulit dicapai oleh algoritma tradisional \cite{jiang2023}. Lebih lanjut, algoritma jalur terpendek seperti Dijkstra membebani \textit{controller} karena harus melakukan kalkulasi ulang secara menyeluruh setiap ada perubahan topologi, sekecil apapun perubahan tersebut. Dampaknya, sering terjadi ketidakseimbangan beban dimana beberapa jalur mengalami kongesti sementara jalur lainnya tidak termanfaatkan dengan baik.

Dalam menentukan jalur yang optimal, ada banyak parameter yang harus dipertimbangkan secara bersamaan, mencakup kinerja perangkat seperti CPU dan RAM, serta kondisi koneksi seperti kualitas \textit{link} dan \textit{bandwidth}. Masing-masing parameter memiliki tingkat kepentingan yang berbeda sesuai dengan kebijakan ISP. Untuk menentukan bobot prioritas antar parameter ini, diperlukan metode pengambilan keputusan multikriteria yang sistematis. \textit{Analytic Hierarchy Process} (AHP) sudah terbukti efektif dalam menangani masalah keputusan dengan banyak kriteria \cite{saaty2008}, khususnya dalam konteks manajemen prioritas jaringan \cite{giouroukelis2026}.

Untuk mengatasi keterbatasan algoritma konvensional dan subjektivitas dalam penentuan parameter, penelitian ini mengajukan penerapan \textit{Graph Attention Network} (GAT) yang dikombinasikan dengan pembobotan AHP. \textit{Graph Neural Networks} (GNN) sudah diakui sebagai metode \textit{state-of-the-art} untuk menangani data berbentuk graf \cite{wu2021}, yang sangat cocok dengan struktur topologi jaringan \cite{zhou2020}. GAT dipilih secara khusus karena memiliki mekanisme \textit{attention} yang dapat memberikan bobot berbeda pada setiap tetangga \textit{node} \cite{velickovic2017}. Kemampuan ini membuat GAT lebih unggul dibanding metode graf konvensional dalam mengenali fitur-fitur penting pada jalur jaringan \cite{kato2024}. Keunggulan utama pendekatan ini terletak pada kecepatan inferensi secara \textit{real-time} tanpa memerlukan komputasi iteratif yang berat.

Meski integrasi GNN dengan optimasi \textit{routing} terbukti dapat meningkatkan ketahanan (\textit{robustness}) jaringan \cite{ye2025}, tantangan utama dalam penerapan AI adalah memastikan validitas data latih agar sesuai dengan preferensi operasional. Karena itu, integrasi AHP dalam penelitian ini berfungsi untuk menghasilkan label kualitas jalur yang valid berdasarkan penilaian pakar \cite{khan2020}. Pendekatan hibrida ini diharapkan dapat mempelajari pola aliran trafik dengan lebih akurat dibanding metode statistik konvensional \cite{rahman2023, lassen2025}, sekaligus menjembatani kesenjangan antara kalkulasi matematis murni dengan intuisi operasional dari pakar jaringan.


\section{Perumusan Masalah}
Berdasarkan uraian latar belakang di atas, beberapa permasalahan yang perlu dijawab dalam penelitian ini adalah:
\begin{enumerate}
    \item Bagaimana menentukan bobot prioritas antar fitur jaringan secara sistematis dan konsisten menggunakan metode AHP untuk menghasilkan label kualitas jalur yang valid?
    
    \item Bagaimana metode GAT dapat menilai dan merekomendasikan jalur terbaik berdasarkan fitur-fitur pada level node dan edge serta bobot prioritas yang diperoleh dari AHP?
    
    \item Bagaimana pendekatan GAT-AHP dapat mengoptimalkan traffic engineering dalam menangani distribusi beban dan pola trafik yang dinamis?
    
    \item Sejauh mana keunggulan model GAT-AHP dibandingkan algoritma konvensional dalam hal adaptabilitas terhadap perubahan topologi dan efisiensi komputasi?
\end{enumerate}

\section{Tujuan Penelitian}
Tujuan yang hendak dicapai melalui penelitian ini meliputi:
\begin{enumerate}
    \item Menerapkan metode AHP untuk mendapatkan bobot prioritas antar fitur jaringan secara valid dan konsisten berdasarkan pertimbangan pakar, yang kemudian digunakan dalam penghitungan label kualitas jalur sebagai target pembelajaran model.
    
    \item Mengembangkan model GAT yang mampu menilai dan merekomendasikan jalur terbaik berdasarkan fitur-fitur node dan edge serta bobot prioritas dari AHP untuk optimasi traffic engineering pada jaringan PT Lare Osing Ndo.
    
    \item Mengukur dan membandingkan performa model GAT-AHP dengan algoritma konvensional dalam hal adaptabilitas terhadap perubahan topologi dinamis dan efisiensi komputasi.
\end{enumerate}

\section{Manfaat Penelitian}
Hasil penelitian ini diharapkan dapat memberikan kontribusi berupa:
\begin{enumerate}
    \item \textbf{Bagi Perusahaan:} Menghadirkan solusi berbasis kecerdasan buatan yang dapat mengoptimalkan traffic engineering, mempercepat respons terhadap gangguan jaringan, dan meningkatkan efisiensi distribusi trafik, sekaligus menjadi landasan bagi pengembangan sistem otomasi jaringan di masa mendatang.
    
    \item \textbf{Secara Teoritis:} Menyumbangkan wawasan ilmiah mengenai integrasi metode AHP dengan GAT untuk optimasi routing dan traffic engineering pada jaringan WAN/ISP skala lokal, serta melengkapi penelitian-penelitian terdahulu yang umumnya lebih menitikberatkan pada pendekatan simulasi atau jaringan dengan skala lebih luas.
\end{enumerate}

\section{Batasan Masalah}
Agar penelitian tetap terarah pada sasaran utama, beberapa pembatasan masalah ditetapkan sebagai berikut:
\begin{enumerate}
    \item Data mengenai topologi dan karakteristik jaringan diperoleh dari perangkat operasional PT Lare Osing Ndo, mencakup fitur pada tingkat \textit{Node} (CPU, RAM, beban Trafik, tingkat Utilisasi) serta fitur pada tingkat \textit{Edge} (Kualitas Link, Bandwidth, Kecepatan, Jarak, dan Status).
    
    \item Penentuan bobot prioritas antar fitur dilakukan melalui metode AHP dengan mendasarkan pada penilaian dari pakar, dalam hal ini \textit{network engineer} berpengalaman di PT Lare Osing Ndo.
    
    \item Pembangunan model rekomendasi jalur menggunakan arsitektur \textit{Graph Attention Network} (GAT) dengan pengaturan \textit{hyperparameter} seperti jumlah lapisan, unit tersembunyi, dan konfigurasi \textit{multi-head attention} yang ditetapkan melalui serangkaian eksperimen guna memperoleh hasil terbaik.
    
    \item Evaluasi model mencakup pengujian pada skenario perubahan topologi dinamis dan perbandingan efisiensi komputasi dengan algoritma pencarian jalur konvensional seperti \textit{Dijkstra}.
    
    \item Dataset yang digunakan untuk pelatihan mencakup sampel jalur dengan tingkat optimalitas yang beragam, dengan proporsi yang disesuaikan agar mendukung proses pembelajaran terawasi (\textit{supervised learning}).
\end{enumerate}