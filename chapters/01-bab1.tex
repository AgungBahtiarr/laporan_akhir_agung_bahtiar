\chapter{PENDAHULUAN}

\section{Latar Belakang}
PT Lare Osing Ndo merupakan \textit{Internet Service Provider} (ISP) lokal yang beroperasi di Banyuwangi dengan infrastruktur berbasis \textit{switch} Layer 2 dan teknologi VLAN. Pada kondisi saat ini (\textit{current state}), manajemen jaringan komunikasi yang kompleks menghadapi tantangan besar dalam menjaga stabilitas layanan. Pendekatan kecerdasan buatan terdistribusi \cite{alhachem2025} mulai diperlukan untuk menangani tantangan tersebut. Tren pertumbuhan trafik dan pengguna internet berdampak langsung pada ISP regional yang harus menjaga \textit{Service Level Agreement} (SLA). Oleh karena itu, menuju era \textit{zero downtime}, penggunaan \textit{machine learning} menjadi krusial untuk memprediksi kegagalan jaringan sebelum terjadi gangguan fatal \cite{basikolo2023}.

Permasalahan utama muncul pada pengelolaan trafik yang semakin rumit akibat penambahan \textit{link} redundansi yang masif. Saat ini, penentuan jalur masih bergantung pada protokol konvensional atau keputusan manual administrator. Pendekatan ini dinilai kurang efektif karena metode \textit{Traffic Engineering} (TE) modern menuntut kemampuan menangani pola trafik dinamis dan heterogen \cite{marouani2024}, hal yang sulit dicapai oleh algoritma tradisional \cite{jiang2023}. Selain itu, algoritma jalur terpendek konvensional (seperti Dijkstra) membebani \textit{controller} karena harus melakukan kalkulasi ulang secara menyeluruh setiap kali terjadi perubahan topologi sekecil apapun. Akibatnya, sering terjadi ketidakseimbangan beban di mana beberapa jalur mengalami kongesti sementara jalur lain \textit{underutilized}.

Dalam menentukan jalur optimal, terdapat banyak parameter yang perlu dipertimbangkan secara bersamaan, meliputi kinerja perangkat (CPU, RAM) dan kondisi koneksi (kualitas link, \textit{bandwidth}). Setiap parameter memiliki tingkat kepentingan yang berbeda menurut kebijakan ISP. Penentuan bobot prioritas antar parameter ini memerlukan metode pengambilan keputusan multikriteria yang sistematis. \textit{Analytic Hierarchy Process} (AHP) telah terbukti efektif dalam menangani masalah pengambilan keputusan dengan banyak kriteria \cite{saaty2008}, terutama dalam konteks manajemen prioritas jaringan \cite{giouroukelis2026}.

Untuk mengatasi permasalahan ketidakefektifan algoritma konvensional dan subjektivitas parameter tersebut, penelitian ini mengusulkan penerapan metode \textit{Graph Attention Network} (GAT) yang diperkuat dengan pembobotan AHP. \textit{Graph Neural Networks} (GNN) telah terbukti sebagai metode yang \textit{state-of-the-art} untuk menangani data berbentuk graf \cite{wu2021}, yang sangat relevan dengan struktur topologi jaringan \cite{zhou2020}. Secara spesifik, GAT dipilih karena memiliki mekanisme \textit{attention} yang mampu memberikan bobot berbeda pada setiap tetangga \textit{node} \cite{velickovic2017}. Kemampuan ini membuat GAT lebih unggul dibandingkan metode graf konvensional dalam mengenali fitur-fitur penting pada jalur jaringan \cite{kato2024}. Keunggulan utama pendekatan ini adalah kecepatan inferensi secara \textit{real-time} tanpa komputasi iteratif yang berat.

Meskipun integrasi GNN dengan optimasi \textit{routing} mampu meningkatkan ketahanan (\textit{robustness}) jaringan \cite{ye2025}, tantangan utama dalam penerapan AI adalah validitas data latih agar sesuai dengan preferensi operasional. Oleh karena itu, integrasi AHP dalam penelitian ini berfungsi untuk menghasilkan label kualitas jalur yang valid berdasarkan penilaian pakar \cite{khan2020}. Pendekatan hibrida ini diharapkan mampu mempelajari pola aliran trafik secara lebih akurat dibandingkan metode statistik biasa \cite{rahman2023, lassen2025}, sekaligus menutup celah kesenjangan antara kalkulasi matematis murni dengan intuisi operasional pakar jaringan.

\section{Perumusan Masalah}
Berdasarkan latar belakang yang telah diuraikan, maka rumusan masalah dalam penelitian ini adalah:
\begin{enumerate}
    \item Bagaimana menerapkan metode \textit{Analytic Hierarchy Process} (AHP) untuk pembobotan parameter kualitas jaringan (kualitas link, bandwidth, kecepatan, jarak, trafik, CPU, dan RAM) sebagai dasar pembentukan label kualitas jalur?
    \item Bagaimana merancang dan mengimplementasikan arsitektur \textit{Graph Attention Network} (GAT) yang mampu mempelajari representasi graf topologi jaringan dan menghasilkan rekomendasi rute optimal?
    \item Bagaimana performa model GAT dengan pembobotan AHP dalam merekomendasikan jalur dibandingkan dengan metode \textit{shortest path} konvensional berdasarkan metrik akurasi, presisi rekomendasi, dan waktu inferensi?
\end{enumerate}

\section{Tujuan Penelitian}
Tujuan dari penelitian ini adalah untuk:
\begin{enumerate}
    \item Mengimplementasikan metode AHP untuk menghasilkan bobot prioritas parameter jaringan yang valid dan konsisten berdasarkan penilaian pakar, yang kemudian digunakan untuk menghitung skor kualitas jalur sebagai label target dalam pelatihan model.
    \item Membangun dan melatih model rekomendasi rute menggunakan arsitektur GAT dengan konfigurasi optimal dan mekanisme \textit{multi-head attention}, yang mampu mempelajari representasi graf topologi jaringan PT Lare Osing Ndo \cite{velickovic2017}.
    \item Mengevaluasi performa model GAT dalam memprediksi jalur optimal dengan mengukur tingkat akurasi prediksi serta membandingkan skor kualitas jalur yang dihasilkan dengan metode konvensional \cite{almasan2022}.
\end{enumerate}

\section{Manfaat Penelitian}
Manfaat yang diharapkan dari penelitian ini adalah:
\begin{enumerate}
    \item \textbf{Bagi Perusahaan:} Memberikan solusi untuk mempercepat penanganan gangguan dan optimasi trafik menggunakan sistem berbasis kecerdasan buatan, serta menyediakan landasan untuk implementasi otomatisasi jaringan di masa depan.
    \item \textbf{Bagi Akademisi:} Memberikan kontribusi ilmiah terkait penerapan algoritma GAT pada manajemen jaringan WAN/ISP lokal, serta memperkaya studi sebelumnya yang lebih banyak berfokus pada simulasi atau jaringan skala besar \cite{marouani2024, saha2024}.
\end{enumerate}

\section{Batasan Masalah}
Agar penelitian lebih terarah dan fokus pada tujuan utama, batasan masalah yang ditetapkan adalah:
\begin{enumerate}
    \item Data topologi dan parameter jaringan diambil dari perangkat aktif milik PT Lare Osing Ndo, yang terdiri dari atribut \textit{Node} (CPU, RAM, Trafik, Utilisasi) dan atribut \textit{Edge} (Kualitas Link, Bandwidth, Kecepatan, Jarak, Status).
    \item Pembobotan prioritas parameter dilakukan menggunakan metode AHP berdasarkan penilaian pakar (\textit{expert judgment}) dari \textit{network engineer} senior PT Lare Osing Ndo.
    \item Model rekomendasi jalur dibangun menggunakan arsitektur \textit{Graph Attention Network} (GAT) dengan konfigurasi \textit{hyperparameter} (seperti jumlah \textit{layer} dan \textit{hidden units}) yang ditentukan melalui proses eksperimen untuk mendapatkan performa terbaik \cite{fey2019}.
    \item Dataset pelatihan terdiri dari sampel jalur optimal dan sub-optimal dengan proporsi yang disesuaikan untuk kebutuhan pembelajaran model (\textit{supervised learning}).
    \item Pelatihan model dilakukan menggunakan mekanisme \textit{early stopping} untuk menghentikan proses pelatihan secara otomatis ketika model telah mencapai konvergensi optimal.
    \item Evaluasi model difokuskan pada ketepatan prediksi skor kualitas jalur (menggunakan metrik seperti MSE) dan akurasi rekomendasi jalur (\textit{Top-k Precision}) dibandingkan dengan metode konvensional.
    \item Sistem yang dibangun berfokus pada memberikan rekomendasi \textit{top-k} jalur terbaik ($k=3$), dan tidak melakukan konfigurasi otomatis (\textit{auto-configuration}) pada perangkat jaringan.
    \item Pengujian ketahanan model dilakukan melalui simulasi skenario kegagalan tautan (\textit{link failure}) dinamis untuk menguji kemampuan adaptasi model (\textit{inductive learning}) terhadap perubahan topologi.
\end{enumerate}
