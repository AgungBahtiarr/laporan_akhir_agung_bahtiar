% ============================================
% ABSTRAK BAHASA INDONESIA
% ============================================

\chapter*{ABSTRAK}
\addcontentsline{toc}{chapter}{ABSTRAK}

\begin{center}
    {\fontsize{14}{16}\bfseries\selectfont
    \MakeUppercase{\judulTA}
    \par}

    \vspace{1cm}

    {\fontsize{12}{14}\selectfont
    \namaMahasiswa\\
    NIM. \nimMahasiswa
    \par}

    \vspace{0.5cm}

    {\fontsize{12}{14}\selectfont
    Pembimbing:\\
    1. \pembimbingSatu\\
    2. \pembimbingDua
    \par}
\end{center}

\vspace{1cm}

\setstretch{1.0}
\setlength{\parindent}{1.25cm}

% ISI ABSTRAK (Maksimal 350 kata)
Abstrak berisi ringkasan komprehensif dari seluruh isi Tugas Akhir. Bagian ini mencakup latar belakang singkat yang menjelaskan konteks penelitian, rumusan masalah atau tujuan penelitian, metode penelitian yang digunakan, hasil utama yang diperoleh, dan kesimpulan penting dari penelitian.

Abstrak harus ditulis dalam satu paragraf dengan spasi 1. Hindari penggunaan singkatan yang tidak umum. Jika terpaksa menggunakan singkatan, berikan kepanjangan lengkap pada penggunaan pertama. Abstrak tidak boleh mengandung referensi ke pustaka, tabel, atau gambar.

Untuk penelitian yang bersifat aplikatif atau pembuatan prototype, jelaskan fungsi utama dan keunggulan dari sistem/produk yang dibuat. Untuk penelitian eksperimental, sebutkan variabel-variabel utama yang diteliti dan temuan signifikan yang diperoleh.

Bagian akhir abstrak harus menyebutkan kontribusi utama atau manfaat praktis dari penelitian ini. Total kata dalam abstrak maksimal 350 kata sesuai dengan pedoman Tugas Akhir Poliwangi.

\vspace{1cm}

\noindent\textbf{Kata kunci:} kata kunci 1, kata kunci 2, kata kunci 3, kata kunci 4, kata kunci 5
\\[0.3cm]
\textit{(Maksimal 5 kata kunci, diurutkan alfabetis)}

\setstretch{1.5}
\clearpage
