\chapter{TINJAUAN PUSTAKA}

\section{Tinjauan Studi}
Tinjauan studi dilakukan untuk memetakan posisi penelitian ini terhadap penelitian-penelitian sebelumnya yang relevan (\textit{State of the Art}). Pemetaan ini bertujuan untuk mengidentifikasi kesenjangan penelitian (\textit{research gap}) dan memastikan kebaruan kontribusi yang ditawarkan.

Penelitian pertama oleh Marouani dkk. (2024) menerapkan \textit{Graph Attention Networks} (GAT) untuk \textit{Traffic Engineering} pada jaringan WAN. Penelitian tersebut membuktikan bahwa mekanisme \textit{attention} pada GAT mampu menangani topologi dinamis lebih baik daripada metode heuristik konvensional \cite{marouani2024}. Namun, penelitian ini hanya berfokus pada metrik performa teknis murni tanpa mempertimbangkan preferensi kebijakan manajemen jaringan yang seringkali bersifat subjektif pada ISP lokal.

Penelitian kedua oleh Almasan dkk. (2022) mengusulkan pendekatan \textit{Deep Reinforcement Learning} (DRL) yang digabungkan dengan GNN untuk optimasi \textit{routing}. Hasilnya menunjukkan kemampuan generalisasi yang kuat terhadap topologi baru \cite{almasan2022}. Meskipun efektif, pendekatan DRL memerlukan lingkungan simulasi yang sangat kompleks dan waktu pelatihan yang lama, yang menjadi kendala jika diterapkan pada infrastruktur ISP skala menengah dengan sumber daya terbatas.

Penelitian ketiga oleh Rahman dan Hasan (2023) menggunakan \textit{Graph Convolutional Network} (GCN) untuk memprediksi aliran trafik \cite{rahman2023}. GCN efektif dalam menangkap fitur spasial, namun memiliki keterbatasan karena memberikan bobot yang seragam pada tetangga \textit{node} (isotropik), sehingga kurang sensitif terhadap variasi kualitas \textit{link} yang signifikan pada jaringan nirkabel atau campuran.

\textbf{Posisi Penelitian:}
Berdasarkan tinjauan di atas, ditemukan celah permasalahan di mana belum ada penelitian yang menggabungkan kemampuan adaptif GAT dalam mempelajari topologi graf dengan metode pembobotan multikriteria (\textit{Analytic Hierarchy Process} - AHP). Integrasi ini penting untuk memastikan bahwa rekomendasi AI tidak hanya optimal secara matematis, tetapi juga valid menurut standar operasional dan intuisi pakar jaringan (\textit{Network Engineer}). Rangkuman perbandingan penelitian ditunjukkan pada Tabel \ref{tab:state_of_the_art}.

\begin{table}[ht]
    \centering
    \caption{State of the Art Penelitian}
    \label{tab:state_of_the_art}
    \small
    \begin{tabular}{p{3cm} p{3cm} p{2.5cm} p{4.5cm}}
        \hline
        \textbf{Peneliti (Tahun)} & \textbf{Masalah} & \textbf{Metode} & \textbf{Hasil \& Perbedaan} \\
        \hline
        Marouani dkk. (2024) \cite{marouani2024} & Optimasi trafik pada WAN & GAT & Unggul di topologi dinamis. \textbf{Bedanya:} Penelitian ini menambahkan validasi pakar via AHP. \\
        \hline
        Almasan dkk. (2022) \cite{almasan2022} & Optimasi \textit{Routing} & DRL + GNN & Generalisasi baik tapi training berat. \textbf{Bedanya:} Penelitian ini menggunakan \textit{Supervised Learning} yang lebih efisien. \\
        \hline
        Rahman \& Hasan (2023) \cite{rahman2023} & Prediksi Trafik & GCN & Efektif spasial. \textbf{Bedanya:} GAT memiliki mekanisme \textit{attention} yang lebih presisi dibanding GCN. \\
        \hline
        \textbf{Penelitian Ini} & \textbf{Rekomendasi Jalur ISP} & \textbf{GAT + AHP} & \textbf{Integrasi \textit{expert judgment} (AHP) ke dalam arsitektur \textit{Deep Learning} (GAT).} \\
        \hline
    \end{tabular}
\end{table}

\section{Landasan Teori}

\subsection{Manajemen Jaringan ISP}
Manajemen jaringan pada \textit{Internet Service Provider} (ISP) melibatkan pemantauan dan pengendalian sumber daya jaringan untuk menjamin ketersediaan layanan. Tantangan utama pada infrastruktur Layer 2 dan VLAN adalah mencegah \textit{looping} dan menyeimbangkan beban trafik \cite{alhachem2025}. Kegagalan dalam manajemen ini dapat berdampak fatal pada SLA (\textit{Service Level Agreement}), sehingga diperlukan metode prediksi kegagalan berbasis \textit{machine learning} untuk menuju era \textit{zero downtime} \cite{basikolo2023}.

\subsection{Graph Neural Networks (GNN)}
\textit{Graph Neural Network} (GNN) adalah kerangka kerja \textit{deep learning} yang dirancang untuk data berbentuk graf. Berbeda dengan CNN yang bekerja pada data grid (gambar), GNN bekerja pada struktur data non-Euclidean. GNN memperbarui representasi fitur sebuah \textit{node} dengan mengagregasi informasi dari tetangganya \cite{wu2021}. Dalam jaringan komputer, perangkat direpresentasikan sebagai \textit{node} dan kabel fisik sebagai \textit{edge} \cite{zhou2020}.

\subsection{Graph Attention Network (GAT)}
\textit{Graph Attention Network} (GAT) adalah pengembangan GNN yang memperkenalkan mekanisme \textit{masked self-attention}. Keunggulan utamanya adalah kemampuan memberikan bobot kepentingan (\textit{attention coefficients}) yang berbeda pada setiap tetangga, tanpa bergantung pada struktur graf statis \cite{velickovic2017}.

Untuk memudahkan pemahaman komputasi, proses perhitungan koefisien atensi $\alpha_{ij}$ dipecah menjadi tiga tahapan sebagai berikut:

\begin{enumerate}
    \item \textbf{Transformasi Linear:} Fitur input node ($\vec{h}$) dikalikan dengan matriks bobot ($W$) yang dipelajari.
    \begin{equation}
        z_i = W \cdot \vec{h}_i
    \end{equation}

    \item \textbf{Mekanisme Atensi:} Menghitung skor kedekatan mentah ($e_{ij}$) antara node sumber dan tetangga menggunakan fungsi aktivasi \textit{LeakyReLU}.
    \begin{equation}
        e_{ij} = \text{LeakyReLU}\left(\vec{a}^T [W\vec{h}_i \, \| \, W\vec{h}_j]\right)
    \end{equation}

    \item \textbf{Normalisasi Softmax:} Mengubah skor mentah menjadi probabilitas atensi (total bernilai 1).
    \begin{equation}
        \alpha_{ij} = \frac{\exp(e_{ij})}{\sum_{k \in \mathcal{N}_i} \exp(e_{ik})}
    \end{equation}
\end{enumerate}

Dimana $\|$ adalah operasi penggabungan (\textit{concatenation}) dan $\vec{a}$ adalah vektor bobot atensi. Mekanisme ini memungkinkan model untuk "memperhatikan" jalur dengan kualitas sinyal lebih baik dan mengabaikan jalur yang buruk \cite{kato2024}.

\subsection{Analytic Hierarchy Process (AHP)}
AHP adalah metode pengambilan keputusan multikriteria yang dikembangkan oleh Thomas L. Saaty. AHP memecah masalah kompleks menjadi hierarki kriteria dan alternatif. Bobot prioritas ditentukan melalui perbandingan berpasangan (\textit{pairwise comparison}) yang divalidasi dengan rasio konsistensi ($CR \le 0.1$) \cite{saaty2008}.

Dalam penelitian ini, AHP digunakan untuk menghitung skor kualitas ($Q$) yang menjadi label target pelatihan:
\begin{equation}
    Q_{path} = \sum w_i \cdot f_i(\text{node}) + \sum w_j \cdot g_j(\text{edge})
\end{equation}
Penggunaan AHP menjamin bahwa model AI belajar dari preferensi pakar yang terukur dan konsisten \cite{khan2020}.

\section{Kerangka Pemikiran}
Kerangka pemikiran menggambarkan alur logika penelitian dari variabel yang diobservasi hingga pengukuran keberhasilan. Sesuai dengan pendekatan sistem cerdas, kerangka ini disusun dalam empat komponen utama: \textit{Indicators}, \textit{Proposed Method}, \textit{Objectives}, dan \textit{Measurement}.

\begin{enumerate}
    \item \textbf{Indicators (Observed Variables):} Merupakan parameter input yang diambil dari perangkat jaringan, terdiri dari parameter Node (CPU, RAM, Trafik, Utilisasi) dan parameter Edge (Kualitas Link, Bandwidth, Kecepatan, Jarak, Status).
    \item \textbf{Proposed Method:} Solusi yang diusulkan menggabungkan dua metode utama. Pertama, \textit{Analytic Hierarchy Process} (AHP) digunakan pada tahap pra-pemrosesan untuk memberikan bobot prioritas valid berdasarkan penilaian pakar. Kedua, \textit{Graph Attention Network} (GAT) digunakan sebagai algoritma pembelajaran utama untuk mengenali pola topologi dan memprediksi kualitas jalur.
    \item \textbf{Objectives:} Tujuan akhir dari sistem adalah menghasilkan model rekomendasi jalur yang memiliki akurasi tinggi dan mampu mengoptimalkan distribusi trafik jaringan.
    \item \textbf{Measurement:} Keberhasilan metode diukur menggunakan metrik statistik \textit{Root Mean Square Error} (RMSE) untuk performa regresi, dan \textit{Accuracy} untuk mengukur ketepatan klasifikasi kualitas jalur yang direkomendasikan.
\end{enumerate}

Visualisasi kerangka pemikiran penelitian ditunjukkan pada Gambar \ref{fig:kerangka_pemikiran}.

\begin{figure}[ht]
    \centering
    \resizebox{\textwidth}{!}{%
    \begin{tikzpicture}[
        node distance=0.5cm,
        box/.style={draw, rounded corners, minimum width=3cm, minimum height=1cm, align=center, font=\small},
        groupbox/.style={draw, rounded corners, inner sep=0.3cm, align=center},
        arrow/.style={->, thick, >=stealth}
    ]

    % --- COLUMN 1: INDICATORS ---
    \node[font=\bfseries] (ind_title) {INDICATORS};

    \node[box, below=0.5cm of ind_title, fill=white] (cpu) {CPU \& RAM\\Usage};
    \node[box, below=0.3cm of cpu, fill=white] (traf) {Interface\\Utilization};
    \node[box, below=0.3cm of traf, fill=white] (link) {Link Quality\\(Signal/Loss)};
    \node[box, below=0.3cm of link, fill=white] (bw) {Bandwidth\\Availability};
    \node[box, below=0.3cm of bw, fill=white] (dist) {Geographical\\Distance};

    % Frame for Indicators
    \node[groupbox, fit=(cpu)(dist), label=below:\textit{Observed Variables}] (ind_group) {};

    % --- COLUMN 2: PROPOSED METHOD ---
    \node[font=\bfseries, right=1.5cm of ind_title] (met_title) {PROPOSED METHOD};

    \node[box, below=0.5cm of met_title, fill=gray!10, minimum width=4cm] (data) {Dataset Generation\\(Topology \& Metrics)};
    \node[box, below=0.8cm of data, fill=gray!10, minimum width=4cm] (ahp) {\textbf{Weighting Strategy}\\Analytic Hierarchy Process\\(AHP)};
    \node[box, below=0.8cm of ahp, fill=gray!20, minimum width=4cm, minimum height=1.5cm] (gat) {\textbf{Learning Algorithm}\\Graph Attention Network\\(Multi-Head Attention)};
    \node[box, below=0.8cm of gat, fill=gray!10, minimum width=4cm] (pred) {Path Quality\\Predictor};

    \node[groupbox, fit=(data)(pred)] (met_group) {};

    % --- COLUMN 3: OBJECTIVES ---
    \node[font=\bfseries, right=1cm of met_title] (obj_title) {OBJECTIVES};

    \node[circle, draw, align=center, minimum size=2.5cm, below=2cm of obj_title] (obj1) {High Accuracy\\Model};
    \node[circle, draw, align=center, minimum size=2.5cm, below=0.5cm of obj1] (obj2) {Optimal Path\\Recommendation};

    % --- COLUMN 4: MEASUREMENT ---
    \node[font=\bfseries, right=1cm of obj_title] (meas_title) {MEASUREMENT};

    \node[box, below=2cm of meas_title] (rmse) {Root Mean\\Square Error (RMSE)};
    \node[box, below=0.5cm of rmse] (acc) {Model\\Accuracy};
    \node[box, below=0.5cm of acc] (time) {Inference Time};

    \node[groupbox, fit=(rmse)(time), label=below:\textit{Observed Results}] (meas_group) {};

    % --- ARROWS ---
    \draw[arrow, dashed] (ind_group.east) -- (met_group.west);
    \draw[arrow] (data) -- (ahp);
    \draw[arrow] (ahp) -- (gat);
    \draw[arrow] (gat) -- (pred);
    \draw[arrow] (met_group.east) -- (obj1.west);
    \draw[arrow] (met_group.east) -- (obj2.west);
    \draw[arrow] (obj1.east) -- (rmse.west);
    \draw[arrow] (obj1.east) -- (acc.west);
    \draw[arrow] (obj2.east) -- (time.west);

    \end{tikzpicture}%
    }
    \caption{Kerangka Pemikiran GAT berbasis AHP}
    \label{fig:kerangka_pemikiran}
\end{figure}
