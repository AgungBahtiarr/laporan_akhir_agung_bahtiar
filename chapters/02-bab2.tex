% ============================================
% BAB 2: TINJAUAN PUSTAKA
% ============================================

\chapter{TINJAUAN PUSTAKA}

Pada bab ini diuraikan teori, temuan dan bahan penelitian sebelumnya yang diperoleh dari berbagai referensi yang dijadikan dasar melakukan penelitian. Kajian pustaka merupakan rangkuman singkat yang komprehensif tentang semua materi terkait. Untuk jenjang Sarjana Terapan, minimal 10 artikel dari 10 tahun terakhir.

\section{Landasan Teori Utama}

Bagian ini menjelaskan teori-teori dasar yang menjadi landasan penelitian. Setiap konsep atau teori harus didukung oleh referensi yang jelas.

\subsection{Konsep Dasar}

Jelaskan konsep dasar pertama yang relevan dengan penelitian Anda. Misalnya, jika penelitian tentang jaringan komputer, jelaskan definisi dan konsep dasar jaringan komputer menurut berbagai ahli \cite{estiasih2015}.

Konsep ini menjadi pondasi untuk memahami pembahasan selanjutnya. Pastikan definisi yang digunakan jelas dan berasal dari sumber yang terpercaya.

\subsection{Teori Pendukung}

Jelaskan teori-teori yang mendukung penelitian. Teori ini harus relevan dengan metode atau pendekatan yang akan digunakan. Gunakan contoh atau ilustrasi jika diperlukan untuk memperjelas pemahaman.

Menurut penelitian \cite{satria2004}, teori X menunjukkan bahwa... Sedangkan penelitian lain \cite{anwar2006} menyatakan bahwa...

\section{Penelitian Terdahulu (State of the Art)}

Bagian ini mengulas penelitian-penelitian sebelumnya yang relevan dengan topik Tugas Akhir. Tujuannya untuk menunjukkan gap atau kesenjangan penelitian yang akan diisi oleh penelitian ini.

\subsection{Penelitian Pertama}

Nama peneliti (tahun) melakukan penelitian tentang... dengan metode... Hasil penelitian menunjukkan bahwa... \cite{nurtjahya2011}. Kelebihan penelitian ini adalah... sedangkan kelemahannya adalah...

\subsection{Penelitian Kedua}

Penelitian lain oleh... (tahun) membahas tentang... Penelitian ini menggunakan pendekatan... dan menghasilkan... \cite{nasoetion2002}.

\subsection{Analisis Gap Penelitian}

Dari beberapa penelitian di atas, dapat diidentifikasi bahwa:
\begin{enumerate}
    \item Penelitian pertama belum membahas aspek X
    \item Penelitian kedua menggunakan metode Y yang memiliki keterbatasan Z
    \item Belum ada penelitian yang mengombinasikan pendekatan A dengan B
\end{enumerate}

Oleh karena itu, penelitian ini akan mengisi gap tersebut dengan cara...

\section{Metode/Teknologi yang Digunakan}

\subsection{Metode A}

Jelaskan metode atau teknologi utama yang akan digunakan dalam penelitian. Berikan dasar teori yang kuat dan referensi yang mendukung.

Metode ini dipilih karena memiliki keunggulan dalam hal... dibandingkan dengan metode lain \cite{pelczar1986}.

\subsection{Metode B (Untuk Sarjana Terapan)}

Untuk jenjang Sarjana Terapan, minimal harus ada perbandingan 2 metode atau lebih. Jelaskan metode kedua yang akan dibandingkan.

Perbandingan kedua metode ini akan membantu dalam pengambilan keputusan solusi terbaik untuk menyelesaikan permasalahan penelitian.

\section{Kerangka Pemikiran}

Bagian ini menjelaskan alur berpikir penelitian dari awal hingga akhir. Dapat disajikan dalam bentuk diagram atau narasi.

\begin{figure}[htbp]
    \centering
    % Uncomment jika sudah ada gambar kerangka pemikiran
    % \includegraphics[width=0.8\textwidth]{images/kerangka-pemikiran.png}
    \caption{Kerangka Pemikiran Penelitian}
    \label{fig:kerangka-pemikiran}
\end{figure}

Berdasarkan landasan teori dan penelitian terdahulu, penelitian ini akan dimulai dengan... kemudian dilanjutkan dengan... sehingga menghasilkan...
