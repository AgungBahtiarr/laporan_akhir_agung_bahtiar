\chapter{METODOLOGI PENELITIAN}

\section{Tahapan Penelitian}
Penelitian ini dilaksanakan melalui serangkaian tahapan sistematis untuk memastikan pencapaian tujuan yang telah ditetapkan. Alur penelitian dimulai dari identifikasi masalah hingga penarikan kesimpulan, sebagaimana divisualisasikan pada Gambar \ref{fig:alur_penelitian}.

\begin{figure}[H]
    \centering
    \includegraphics[width=10cm]{images/flowchart3.png} 
    % Pastikan file gambar flowchart3.png ada di folder images
    \caption{Diagram Alur Penelitian}
    \label{fig:alur_penelitian}
\end{figure}

Secara rinci, tahapan penelitian dijelaskan sebagai berikut:

\subsection{Studi Literatur}
Tahap ini difokuskan pada pemahaman konsep dasar melalui jurnal internasional bereputasi dan buku referensi. Topik utama meliputi \textit{Graph Attention Network} (GAT) untuk data topologi \cite{velickovic2017}, metode \textit{Analytic Hierarchy Process} (AHP) untuk pembobotan multikriteria \cite{saaty2008}, serta teknik \textit{Traffic Engineering} modern pada jaringan ISP \cite{marouani2024}.

\subsection{Pengumpulan Data}
Pengumpulan data dilakukan untuk mendapatkan data topologi jaringan riil dan preferensi kebijakan routing. Instrumen yang digunakan meliputi:
\begin{itemize}
    \item \textbf{Observasi SNMP:} Mengambil data metrik perangkat dan link dari perangkat aktif PT Lare Osing Ndo secara \textit{real-time}.
    \item \textbf{Kuesioner Pakar:} Mengumpulkan penilaian subjektif dari \textit{Network Engineer} senior untuk mengisi matriks perbandingan berpasangan AHP.
\end{itemize}

\subsection{Preprocessing dan AHP}
Data mentah dari SNMP dinormalisasi menggunakan teknik \textit{Min-Max Scaling}. Selanjutnya, bobot prioritas kriteria dihitung menggunakan metode AHP. Validitas data diuji menggunakan \textit{Consistency Ratio} (CR). Jika nilai $CR > 0.1$, pakar diminta meninjau ulang penilaiannya untuk menjaga konsistensi data \cite{bose2024}.

\subsection{Pengembangan Model GAT}
Model GAT dibangun menggunakan kerangka kerja \textit{PyTorch Geometric}. Model dilatih menggunakan dataset jalur yang telah dilabeli dengan skor kualitas hasil perhitungan AHP. Strategi pelatihan melibatkan skenario kegagalan \textit{link} (\textit{failure scenarios}) untuk meningkatkan ketahanan (\textit{robustness}) model terhadap perubahan topologi.

\subsection{Evaluasi dan Analisis}
Performa model diuji menggunakan data validasi yang belum pernah dilihat sebelumnya (20\% dari dataset). Metrik evaluasi utama adalah \textit{Mean Squared Error} (MSE) untuk regresi skor dan Akurasi prediksi. Analisis dilakukan dengan membandingkan rekomendasi jalur model terhadap jalur terpendek konvensional (Dijkstra) dan jalur optimal manual.

\section{Metode Pengumpulan Data}
Data penelitian diklasifikasikan menjadi data primer yang bersumber langsung dari infrastruktur jaringan PT Lare Osing Ndo.

\subsection{Parameter Jaringan (SNMP)}
Data kuantitatif diambil menggunakan protokol SNMP. Agar representasi graf sesuai dengan struktur data pelatihan model GAT, parameter dibagi menjadi dimensi Node dan Edge sebagai berikut:

\begin{enumerate}
    \item \textbf{Atribut Node (5 Dimensi):}
    \begin{itemize}
        \item \textit{CPU Usage}: Persentase penggunaan prosesor perangkat.
        \item \textit{RAM Usage}: Persentase penggunaan memori perangkat.
        \item \textit{Traffic In}: Total trafik masuk rata-rata.
        \item \textit{Traffic Out}: Total trafik keluar rata-rata.
        \item \textit{Interface Utilization}: Rata-rata penggunaan antarmuka pada node.
    \end{itemize}
    
    \item \textbf{Atribut Edge (7 Dimensi):}
    \begin{itemize}
        \item \textit{Link Quality}: Kualitas sinyal atau kesehatan link (dinormalisasi).
        \item \textit{Bandwidth Utilization Side A}: Persentase penggunaan bandwidth dari sisi perangkat sumber.
        \item \textit{Bandwidth Utilization Side B}: Persentase penggunaan bandwidth dari sisi perangkat tujuan.
        \item \textit{Interface Speed Side A}: Kecepatan fisik port pada sisi sumber.
        \item \textit{Interface Speed Side B}: Kecepatan fisik port pada sisi tujuan.
        \item \textit{Geographical Distance}: Jarak fisik antar perangkat (km).
        \item \textit{Link Status}: Status operasional link (Up/Down).
    \end{itemize}
\end{enumerate}

\subsection{Penilaian Pakar (AHP)}
Data kualitatif diperoleh melalui kuesioner AHP. Responden (\textit{Network Engineer} Senior) diminta membandingkan tingkat kepentingan antar parameter (misal: "Apakah Kualitas Link lebih penting daripada Jarak?"). Skala penilaian menggunakan Skala Saaty (1-9).

\section{Preprocessing Data}

\subsection{Pembersihan dan Normalisasi}
Data mentah seringkali memiliki rentang nilai yang berbeda jauh. Normalisasi dilakukan dengan rumus:
\begin{equation}
    x' = \frac{x - \min(x)}{\max(x) - \min(x)}
\end{equation}
Tujuannya adalah membawa seluruh fitur ke dalam rentang [0, 1] agar proses pelatihan GAT berjalan stabil dan konvergen lebih cepat.

\subsection{Konversi ke Struktur Graf}
Data tabular dikonversi menjadi objek Graf \textit{PyTorch Geometric} yang terdiri dari:
\begin{itemize}
    \item \texttt{x}: Matriks fitur node berdimensi $[N \times 5]$ (CPU, RAM, Util, Temp, Status).
    \item \texttt{edge\_index}: Matriks konektivitas (indeks \textit{source} $\to$ \textit{target}).
    \item \texttt{edge\_attr}: Matriks fitur edge berdimensi $[E \times 7]$ (Signal, BW, Dist, Latency, Jitter, Loss, Cost).
\end{itemize}

\section{Perhitungan Bobot AHP}
Bobot AHP digunakan untuk menghasilkan \textit{"Ground Truth"} atau label kualitas jalur yang akan dipelajari oleh AI.

\subsection{Matriks Perbandingan Berpasangan}
Pakar menyusun matriks $A$ berukuran $n \times n$. Bobot prioritas $w$ diperoleh dengan mencari vektor eigen utama dari persamaan:
\begin{equation}
    A \cdot w = \lambda_{max} \cdot w
\end{equation}

\subsection{Uji Konsistensi}
Konsistensi dihitung dengan indeks $CI = (\lambda_{max} - n)/(n-1)$ dan rasio $CR = CI/RI$. Nilai $CR \le 0.10$ menjadi syarat mutlak agar bobot AHP dapat digunakan dalam pelabelan dataset.

\section{Pembentukan Dataset Training}
Dataset pelatihan terdiri dari 10.000 sampel jalur yang dibagi menjadi dua kategori untuk memastikan keseimbangan data:
\begin{enumerate}
    \item \textbf{Jalur Optimal (60\%):} Jalur terbaik yang dihasilkan oleh algoritma Dijkstra menggunakan \textit{cost} dari bobot AHP. Ini mengajarkan model tentang karakteristik jalur yang "ideal".
    \item \textbf{Jalur Sub-optimal Valid (40\%):} Jalur alternatif yang didapatkan dari algoritma \textit{K-Shortest Path} (urutan ke-2, ke-3, dst). Ini berfungsi sebagai \textit{negative sample} agar model bisa membedakan jalur terbaik dengan jalur yang "sekadar cukup".
\end{enumerate}

\section{Arsitektur Model (Proposed Method)}
Sistem yang diusulkan menggunakan arsitektur \textit{Graph Attention Network} (GAT) dengan spesifikasi detil pada Tabel \ref{tab:model_spec}.

\begin{table}[H]
    \centering
    \caption{Spesifikasi Arsitektur GAT}
    \label{tab:model_spec}
    \small
    \renewcommand{\arraystretch}{1.2} 
    \begin{tabular}{p{3cm} p{5cm} p{6cm}}
        \toprule
        \textbf{Komponen} & \textbf{Spesifikasi} & \textbf{Fungsi} \\
        \midrule
        Input Layer & Node (5 dim), Edge (7 dim) & Menerima fitur jaringan lengkap yang telah dinormalisasi. \\
        \midrule
        GAT Layer 1 & 64 Hidden Units, 4 Heads & Ekstraksi fitur graf tingkat rendah (lokal). \\
        \midrule
        GAT Layer 2 & 64 Hidden Units, 4 Heads & Ekstraksi fitur graf tingkat menengah (tetangga). \\
        \midrule
        GAT Layer 3 & 64 Hidden Units, 1 Head & Agregasi akhir menjadi \textit{Node Embeddings}. \\
        \midrule
        % Perbaikan di sini: Menggunakan math mode ($\to$) dengan spasi yang cukup
        Path Predictor & MLP (128 $\to$ 64 $\to$ 1) & Memprediksi skor kualitas jalur dari gabungan embedding node. \\
        \bottomrule
    \end{tabular}
\end{table}

Mekanisme \textit{Multi-Head Attention} memungkinkan model mempelajari berbagai aspek hubungan antar perangkat secara paralel \cite{velickovic2017}.

\section{Skenario Pengujian (Robustness)}
Untuk menjawab tantangan jaringan dinamis, penelitian ini merancang 30 skenario kegagalan (\textit{Failure Scenarios}).
\begin{itemize}
    \item \textbf{Metode:} Menonaktifkan 10\% hingga 30\% \textit{link} secara acak pada graf topologi.
    \item \textbf{Tujuan:} Menguji kemampuan \textit{inductive learning} model, yaitu apakah model tetap mampu menemukan jalur alternatif terbaik pada topologi yang belum pernah dilihat sebelumnya (tanpa \textit{retraining}) \cite{ye2025}.
\end{itemize}

\section{Evaluasi Model}
Keberhasilan model diukur berdasarkan indikator berikut:
\begin{enumerate}
    \item \textbf{Loss Function (MSE):} Mengukur selisih kuadrat antara skor prediksi model dan skor AHP. Target: $Loss < 0.01$.
    \item \textbf{Akurasi ($\pm$ 0.08):} Persentase prediksi yang memiliki deviasi kurang dari 8\% terhadap target. Target: $Akurasi > 80\%$.
    \item \textbf{Waktu Inferensi:} Waktu yang dibutuhkan untuk memberikan rekomendasi jalur. Target: $< 100$ milidetik.
\end{enumerate}

\section{Lingkungan Implementasi}
Eksperimen dilakukan menggunakan perangkat keras dengan prosesor Intel Core i5/Ryzen 5 dan RAM 16GB. Perangkat lunak yang digunakan berbasis Python 3.9 dengan pustaka utama \textit{PyTorch Geometric} untuk GNN dan \textit{NetworkX} untuk manipulasi graf dasar.