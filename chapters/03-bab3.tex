% ============================================
% BAB 3: METODOLOGI PENELITIAN
% ============================================

\chapter{METODOLOGI PENELITIAN}

Pada bab ini diuraikan desain, metode atau pendekatan yang digunakan dalam menjawab permasalahan penelitian untuk mencapai tujuan penelitian serta tahapan penelitian secara rinci, singkat dan jelas.

\section{Desain Penelitian}

Jelaskan desain atau rancangan penelitian secara umum. Apakah penelitian ini bersifat deskriptif, eksperimental, komparatif, atau pengembangan sistem/produk.

Penelitian ini menggunakan desain... dengan pendekatan... Alasan pemilihan desain ini adalah...

\section{Lokasi dan Waktu Penelitian}

Penelitian dilaksanakan di... pada periode... Pemilihan lokasi ini didasarkan pada pertimbangan...

\subsection{Lokasi Penelitian}
Jelaskan lokasi penelitian dengan spesifik. Jika di industri/perusahaan, sebutkan nama dan alamatnya.

\subsection{Waktu Penelitian}
Penelitian dilaksanakan selama... bulan, mulai dari... hingga...

\section{Objek Penelitian}

Jelaskan objek atau subjek penelitian secara detail. Untuk Diploma III, objek cenderung sederhana (satu komponen). Untuk Sarjana Terapan, objek lebih kompleks (suatu sistem).

\section{Variabel Penelitian}

\subsection{Variabel Independen}
Variabel bebas yang akan dimanipulasi atau diubah-ubah: ...

\subsection{Variabel Dependen}
Variabel terikat yang akan diamati atau diukur: ...

\subsection{Variabel Kontrol}
Variabel yang dijaga tetap: ...

\section{Instrumen Penelitian}

Jelaskan alat, bahan, software, hardware yang digunakan dalam penelitian:

\subsection{Perangkat Keras}
\begin{itemize}
    \item Komputer dengan spesifikasi...
    \item Sensor/alat ukur...
    \item Perangkat lain...
\end{itemize}

\subsection{Perangkat Lunak}
\begin{itemize}
    \item Software A versi...
    \item Software B versi...
    \item Library/framework...
\end{itemize}

\subsection{Bahan Penelitian}
Jika penelitian melibatkan bahan tertentu, sebutkan spesifikasinya.

\section{Metode Pengumpulan Data}

\subsection{Data Primer}
Data primer diperoleh melalui:
\begin{enumerate}
    \item Observasi langsung...
    \item Wawancara dengan...
    \item Eksperimen/pengujian...
    \item Kuesioner (jika menggunakan kuesioner, lampirkan di Lampiran)
\end{enumerate}

\subsection{Data Sekunder}
Data sekunder diperoleh dari:
\begin{enumerate}
    \item Dokumen perusahaan...
    \item Literatur jurnal...
    \item Database online...
\end{enumerate}

\section{Metode Analisis Data}

Jelaskan metode atau teknik analisis yang digunakan untuk mengolah data:

\subsection{Analisis Deskriptif}
Untuk Diploma III, analisis dapat menggunakan konsep sederhana...

\subsection{Analisis Statistik (Untuk Sarjana Terapan)}
Untuk Sarjana Terapan, gunakan analisis statistik inferensi atau simulasi. Misalnya:
\begin{itemize}
    \item Uji normalitas
    \item Uji hipotesis
    \item Analisis regresi
    \item Analisis varian (ANOVA)
\end{itemize}

\subsection{Perbandingan Metode (Untuk Sarjana Terapan)}
Jelaskan cara membandingkan dua metode atau lebih. Parameter perbandingan yang digunakan: ...

\section{Tahapan Penelitian}

Jelaskan langkah-langkah penelitian secara berurutan dan sistematis.

\subsection{Tahap Persiapan}
\begin{enumerate}
    \item Studi literatur
    \item Identifikasi masalah
    \item Perumusan masalah
    \item Pengajuan proposal
\end{enumerate}

\subsection{Tahap Pelaksanaan}
\begin{enumerate}
    \item Perancangan sistem/alat/metode
    \item Implementasi
    \item Pengumpulan data
    \item Pengujian
    \item Analisis data
\end{enumerate}

\subsection{Tahap Penyelesaian}
\begin{enumerate}
    \item Evaluasi hasil
    \item Penyusunan kesimpulan
    \item Dokumentasi laporan
\end{enumerate}

\section{Diagram Alir Penelitian}

\begin{figure}[htbp]
    \centering
    % Uncomment dan ganti dengan diagram alir Anda
    % \includegraphics[width=0.6\textwidth]{images/flowchart-penelitian.png}
    \caption{Diagram Alir Penelitian}
    \label{fig:flowchart}
\end{figure}

Gambar \ref{fig:flowchart} menunjukkan alur penelitian secara keseluruhan dari tahap awal hingga akhir.

\section{Jadwal Penelitian}

\begin{table}[htbp]
    \caption{Jadwal Kegiatan Penelitian}
    \label{tab:jadwal}
    \centering
    \begin{tabular}{|l|c|c|c|c|c|c|}
        \hline
        \multirow{2}{*}{\textbf{Kegiatan}} & \multicolumn{6}{c|}{\textbf{Bulan Ke-}} \\
        \cline{2-7}
        & \textbf{1} & \textbf{2} & \textbf{3} & \textbf{4} & \textbf{5} & \textbf{6} \\
        \hline
        Studi Literatur & X & X & & & & \\
        \hline
        Perancangan & & X & X & & & \\
        \hline
        Implementasi & & & X & X & & \\
        \hline
        Pengujian & & & & X & X & \\
        \hline
        Analisis Data & & & & & X & X \\
        \hline
        Penyusunan Laporan & & & & & X & X \\
        \hline
    \end{tabular}
\end{table}

Tabel \ref{tab:jadwal} menunjukkan jadwal pelaksanaan penelitian selama 6 bulan.
