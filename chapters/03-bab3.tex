\chapter{METODOLOGI PENELITIAN}

\section{Tahapan Penelitian}
Penelitian ini dilaksanakan melalui serangkaian tahapan sistematis untuk memastikan pencapaian tujuan yang telah ditetapkan. Alur penelitian dimulai dari studi literatur, pengumpulan data, hingga evaluasi model, sebagaimana divisualisasikan pada Gambar \ref{fig:alur_penelitian}.

\begin{figure}[H]
    \centering
    \includegraphics[width=0.5\textwidth]{images/flowchart3.png}
    \caption{Diagram Alur Penelitian}
    \label{fig:alur_penelitian}
\end{figure}

Secara rinci, tahapan penelitian dijelaskan dalam sub-bab berikut.

\section{Metode Pengumpulan dan Pengolahan Data}
Data penelitian diklasifikasikan menjadi data primer yang bersumber langsung dari infrastruktur jaringan PT Lare Osing Ndo. Proses ini mencakup akuisisi data mentah hingga pembobotan fitur.

\subsection{Parameter Jaringan (SNMP)}
Data kuantitatif diambil menggunakan protokol SNMP (\textit{Simple Network Management Protocol}). Agar representasi graf sesuai dengan struktur data pelatihan model GAT, parameter dibagi menjadi dimensi \textit{Node} dan \textit{Edge} sebagai berikut:

\begin{enumerate}
    \item \textbf{Node Features}: Parameter yang berkaitan dengan perangkat keras, seperti persentase penggunaan CPU, penggunaan memori (RAM), dan utilisasi antarmuka.
    \item \textbf{Edge Features}: Parameter yang berkaitan dengan kualitas koneksi antar perangkat, seperti \textit{bandwidth availability}, \textit{latency}, dan kualitas sinyal.
\end{enumerate}

\subsection{Pembobotan AHP dan Uji Konsistensi}
Untuk menangani subjektivitas dalam penentuan prioritas parameter, pakar jaringan diminta menilai tingkat kepentingan relatif antar kriteria menggunakan Skala Saaty (1--9) \cite{saaty2008}. Penilaian tersebut disusun ke dalam matriks perbandingan berpasangan $A$ berukuran $n \times n$.

Bobot prioritas $w$ (vektor eigen) dihitung melalui langkah-langkah berikut:
\begin{enumerate}
    \item Melakukan normalisasi pada setiap kolom matriks $A$.
    \item Menghitung nilai rata-rata dari setiap baris matriks yang telah dinormalisasi untuk mendapatkan vektor bobot prioritas $w$.
\end{enumerate}

Secara matematis, hal ini setara dengan menyelesaikan persamaan eigen:
\begin{equation}
    A \cdot w = \lambda_{max} \cdot w
\end{equation}
Dimana $\lambda_{max}$ adalah nilai eigen maksimum.

Selanjutnya, karena manusia memiliki keterbatasan dalam menjaga konsistensi, validitas bobot diuji menggunakan rasio konsistensi ($CR$).
\begin{equation}
    CR = \frac{CI}{RI}
\end{equation}
Dimana $CI$ adalah \textit{Consistency Index} dan $RI$ adalah \textit{Random Index}. Jika nilai $CR \le 0.1$, maka bobot dianggap valid dan dapat digunakan untuk pelabelan data.

\section{Pembentukan Dataset Training}
Kualitas model sangat bergantung pada kualitas data latih. Dataset disusun untuk merepresentasikan berbagai kondisi jaringan, mulai dari kondisi ideal hingga kondisi padat (kongesti).

\subsection{Pembangkitan Label Kualitas (Ground Truth)}
Setiap jalur yang mungkin dalam topologi diberikan label skor kualitas ($Q$) yang dihitung berdasarkan penjumlahan terbobot dari nilai atribut ($f$) dikalikan dengan bobot AHP ($w$):
\begin{equation}
    Q_{path} = \sum_{i=1}^{k} w_i \cdot f_i
\end{equation}
Semakin tinggi nilai $Q$, semakin baik kualitas jalur tersebut menurut preferensi pakar.

\subsection{Kategori Sampel Jalur}
Agar model dapat membedakan kualitas rute, sampel data dibagi menjadi dua kategori:
\begin{itemize}
    \item \textbf{Jalur Optimal:} Jalur dengan skor $Q$ tertinggi yang dihasilkan oleh algoritma Dijkstra menggunakan bobot biaya dari AHP.
    \item \textbf{Jalur Sub-optimal Valid:} Jalur alternatif (urutan ke-2, ke-3, dst) yang dihasilkan oleh algoritma \textit{K-Shortest Path}, berfungsi sebagai pembanding negatif agar model mempelajari nuansa perbedaan kualitas.
\end{itemize}

\section{Rancangan Arsitektur Model (Proposed Method)}
Sistem yang diusulkan menggunakan arsitektur \textit{Graph Attention Network} (GAT) sebagaimana diperkenalkan oleh Veličković et al. \cite{velickovic2017}. Spesifikasi rancangan awal (\textit{baseline}) ditunjukkan pada Tabel \ref{tab:model_spec}, namun parameter final akan ditentukan melalui proses \textit{hyperparameter tuning}.

\begin{table}[H]
    \centering
    \caption{Rancangan Awal Arsitektur GAT}
    \label{tab:model_spec}
    \small
    \renewcommand{\arraystretch}{1.2}
    \begin{tabular}{p{3cm} p{5cm} p{5cm}}
        \toprule
        \textbf{Komponen} & \textbf{Konfigurasi Awal} & \textbf{Fungsi} \\
        \midrule
        Input Layer & Node Features, Edge Features & Menerima fitur jaringan lengkap yang telah dinormalisasi. \\
        \midrule
        GAT Layers & Stacked Layers (e.g., 3 Layers), Multi-Head Attention & Ekstraksi fitur graf dari level lokal hingga tetangga yang lebih luas. \\
        \midrule
        Hidden Units & Tunable (e.g., 64/128 units) & Dimensi fitur tersembunyi untuk merepresentasikan kompleksitas data. \\
        \midrule
        Path Predictor & Multilayer Perceptron (MLP) & Memprediksi skor kualitas jalur dari gabungan embedding node (agregasi). \\
        \bottomrule
    \end{tabular}
\end{table}

\section{Skenario Pengujian (Robustness)}
Untuk menjawab tantangan jaringan dinamis, penelitian ini merancang simulasi kegagalan tautan (\textit{Failure Scenarios}).
\begin{itemize}
    \item \textbf{Metode:} Menonaktifkan sejumlah persentase \textit{link} secara acak pada graf topologi saat pengujian.
    \item \textbf{Tujuan:} Menguji kemampuan \textit{inductive learning} model, yaitu memastikan model tetap mampu menemukan jalur alternatif terbaik meskipun topologi berubah drastis dibanding saat pelatihan.
\end{itemize}

\section{Evaluasi Model}
Keberhasilan model diukur berdasarkan indikator kinerja utama berikut:
\begin{enumerate}
    \item \textbf{Loss Function (MSE):} Mengukur selisih kuadrat antara skor prediksi model dan skor referensi AHP, dengan tujuan meminimalkan nilai \textit{error} saat konvergensi.
    \item \textbf{Akurasi Prediksi:} Tingkat ketepatan model dalam merekomendasikan jalur yang sesuai dengan preferensi pakar (\textit{Top-k Precision}, dengan $k=3$).
    \item \textbf{Waktu Inferensi:} Waktu komputasi yang dibutuhkan model untuk memberikan rekomendasi jalur, dengan target performa \textit{real-time} (< 100 ms).
\end{enumerate}
