% ============================================
% BAB 5: PENUTUP
% ============================================

\chapter{PENUTUP}

\section{Kesimpulan}

Kesimpulan harus menjawab tujuan penelitian yang telah ditetapkan pada Bab 1. Kesimpulan dibuat berdasarkan hasil penelitian dan pembahasan pada Bab 4.

Berdasarkan hasil penelitian dan pembahasan yang telah dilakukan, dapat disimpulkan bahwa:

\begin{enumerate}
    \item Sistem yang dikembangkan berhasil mencapai tujuan penelitian dengan akurasi 92.5\%, melebihi target yang ditetapkan sebesar 90\%.

    \item Perbandingan antara Metode A dan Metode B menunjukkan bahwa Metode A lebih unggul dalam hal akurasi (92.5\% vs 89.3\%) dan waktu komputasi (3.2 detik vs 4.5 detik).

    \item Implementasi sistem mampu meningkatkan efisiensi proses sebesar 88.3\%, sehingga dapat memberikan manfaat praktis bagi pengguna.

    \item Faktor-faktor yang mempengaruhi kinerja sistem adalah kualitas data input, parameter yang digunakan, dan spesifikasi hardware.
\end{enumerate}

\textit{Catatan: Sesuaikan jumlah dan isi kesimpulan dengan tujuan penelitian Anda}

\section{Saran}

Saran dibuat berdasarkan keterbatasan penelitian dan rekomendasi untuk penelitian selanjutnya. Ada dua jenis saran:

\subsection{Saran untuk Penelitian Selanjutnya}

Untuk penelitian selanjutnya, disarankan:

\begin{enumerate}
    \item Mengembangkan sistem dengan menambahkan fitur X agar dapat menangani kasus yang lebih kompleks.

    \item Melakukan pengujian dengan dataset yang lebih besar dan beragam untuk meningkatkan generalisasi model.

    \item Mengeksplorasi metode lain seperti Metode C atau kombinasi beberapa metode untuk meningkatkan akurasi.

    \item Mengoptimalkan algoritma agar dapat berjalan pada perangkat dengan spesifikasi hardware yang lebih rendah.

    \item Melakukan studi komparatif yang lebih mendalam dengan melibatkan lebih banyak metode pembanding.
\end{enumerate}

\subsection{Saran untuk Implementasi Praktis}

Untuk implementasi sistem di lapangan, disarankan:

\begin{enumerate}
    \item Melakukan pelatihan kepada pengguna agar dapat mengoperasikan sistem dengan optimal.

    \item Menyediakan dokumentasi lengkap dan panduan penggunaan yang mudah dipahami.

    \item Melakukan maintenance rutin untuk menjaga performa sistem tetap optimal.

    \item Mengembangkan sistem feedback dari pengguna untuk perbaikan berkelanjutan.

    \item Mempertimbangkan aspek keamanan data dan privasi dalam implementasi sistem.
\end{enumerate}

\textit{Catatan: Pilih salah satu atau kedua jenis saran sesuai dengan jenis penelitian Anda}
