% ============================================
% BAB 4: HASIL DAN PEMBAHASAN
% ============================================

\chapter{HASIL DAN PEMBAHASAN}

Bab ini memuat hasil penelitian dan pembahasannya. Ditulis secara terpadu pada tiap topik pembahasan. Penyajian hasil dan pembahasan dapat disertai tabel maupun gambar, sesuai dengan kebutuhan penjelasan permasalahan tersebut.

\section{Hasil Penelitian}

\subsection{Hasil Perancangan/Implementasi}

Jelaskan hasil dari tahap perancangan atau implementasi sistem/produk/metode yang dikembangkan. Sertakan gambar, diagram, atau flowchart untuk memperjelas.

\begin{figure}[htbp]
    \centering
    % \includegraphics[width=0.8\textwidth]{images/hasil-perancangan.png}
    \caption{Hasil Perancangan Sistem}
    \label{fig:hasil-perancangan}
\end{figure}

Gambar \ref{fig:hasil-perancangan} menunjukkan...

\subsection{Hasil Pengujian}

Jelaskan hasil pengujian yang dilakukan. Sajikan dalam bentuk tabel atau grafik untuk memudahkan analisis.

\begin{table}[htbp]
    \caption{Hasil Pengujian Sistem}
    \label{tab:hasil-pengujian}
    \centering
    \begin{tabular}{|l|c|c|c|}
        \hline
        \textbf{No} & \textbf{Parameter} & \textbf{Target} & \textbf{Hasil} \\
        \hline
        1 & Akurasi & > 90\% & 92.5\% \\
        \hline
        2 & Waktu Proses & < 5 detik & 3.2 detik \\
        \hline
        3 & Efisiensi & > 85\% & 88.3\% \\
        \hline
    \end{tabular}
\end{table}

Berdasarkan Tabel \ref{tab:hasil-pengujian}, dapat dilihat bahwa...

\subsection{Hasil Analisis Data}

Jelaskan hasil analisis data yang diperoleh. Untuk Sarjana Terapan, gunakan analisis statistik inferensi.

\section{Pembahasan}

\subsection{Pembahasan Hasil Perancangan}

Bahas secara mendalam hasil perancangan dengan membandingkannya dengan teori atau penelitian sebelumnya. Jelaskan mengapa hasil tersebut diperoleh.

Hasil perancangan menunjukkan bahwa sistem yang dikembangkan mampu... Hal ini sesuai dengan penelitian yang dilakukan oleh... yang menyatakan bahwa...

\subsection{Pembahasan Hasil Pengujian}

Bahas hasil pengujian secara detail. Bandingkan dengan penelitian terdahulu dan teori yang ada.

Dari hasil pengujian yang dilakukan, diperoleh akurasi sebesar 92.5\%. Hasil ini lebih tinggi dibandingkan penelitian... yang memperoleh akurasi 88\%. Peningkatan ini disebabkan oleh...

\subsection{Perbandingan Metode (Untuk Sarjana Terapan)}

Untuk jenjang Sarjana Terapan, wajib ada perbandingan minimal 2 metode.

\begin{table}[htbp]
    \caption{Perbandingan Metode A dan Metode B}
    \label{tab:perbandingan-metode}
    \centering
    \begin{tabular}{|l|c|c|}
        \hline
        \textbf{Kriteria} & \textbf{Metode A} & \textbf{Metode B} \\
        \hline
        Akurasi & 92.5\% & 89.3\% \\
        \hline
        Waktu Komputasi & 3.2 detik & 4.5 detik \\
        \hline
        Kompleksitas & Sedang & Tinggi \\
        \hline
    \end{tabular}
\end{table}

Berdasarkan Tabel \ref{tab:perbandingan-metode}, Metode A lebih unggul dalam hal akurasi dan waktu komputasi, sehingga dipilih sebagai solusi terbaik untuk kasus ini.

\subsection{Kelebihan dan Kekurangan}

Jelaskan kelebihan dan kekurangan dari sistem/metode yang dikembangkan.

\textbf{Kelebihan:}
\begin{itemize}
    \item Mampu menghasilkan akurasi tinggi
    \item Waktu proses yang cepat
    \item User interface yang mudah digunakan
\end{itemize}

\textbf{Kekurangan:}
\begin{itemize}
    \item Masih memerlukan dataset yang besar
    \item Belum bisa menangani kasus tertentu
    \item Memerlukan spesifikasi hardware yang memadai
\end{itemize}

\section{Implikasi Hasil Penelitian}

Jelaskan implikasi atau dampak dari hasil penelitian terhadap praktik atau teori yang ada.

Hasil penelitian ini memberikan kontribusi dalam hal... dan dapat diaplikasikan untuk...
