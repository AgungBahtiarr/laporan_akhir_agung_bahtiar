% ============================================
% ABSTRACT (ENGLISH)
% ============================================

\chapter*{ABSTRACT}
\addcontentsline{toc}{chapter}{ABSTRACT}

\begin{center}
    {\fontsize{14}{16}\bfseries\selectfont
    \MakeUppercase{\judulTA}
    \par}

    \vspace{1cm}

    {\fontsize{12}{14}\selectfont
    \namaMahasiswa\\
    Student ID. \nimMahasiswa
    \par}

    \vspace{0.5cm}

    {\fontsize{12}{14}\selectfont
    Supervisors:\\
    1. \pembimbingSatu\\
    2. \pembimbingDua
    \par}
\end{center}

\vspace{1cm}

\setstretch{1.0}
\setlength{\parindent}{1.25cm}

% ABSTRACT CONTENT (Maximum 350 words)
The abstract contains a comprehensive summary of the entire Final Project. This section includes a brief background explaining the research context, problem formulation or research objectives, research methods used, main results obtained, and important conclusions from the research.

The abstract must be written in one paragraph with single spacing. Avoid using uncommon abbreviations. If it is necessary to use abbreviations, provide the full version at the first use. The abstract should not contain references to literature, tables, or images.

For research that is applicative or prototype making, explain the main functions and advantages of the system/product created. For experimental research, mention the main variables studied and significant findings obtained.

The final part of the abstract should mention the main contribution or practical benefits of this research. The total words in the abstract are a maximum of 350 words according to the Poliwangi Final Project guidelines.

\vspace{1cm}

\noindent\textbf{Keywords:} keyword 1, keyword 2, keyword 3, keyword 4, keyword 5
\\[0.3cm]
\textit{(Maximum 5 keywords, sorted alphabetically)}

\setstretch{1.5}
\clearpage
