\chapter*{ABSTRAK} % Pakai bintang agar tidak ada nomor Bab
\addcontentsline{toc}{chapter}{ABSTRAK}

\begin{center}
    \textbf{JUDUL SKRIPSI ANDA DISINI}
    \vspace{1em}

    Nama Mahasiswa\\
    NIM: ......
\end{center}

\vspace{1em}
\begin{singlespace} % Spasi 1.0 [cite: 778]
    % Paragraf masuk 1 tab (1.25cm) [cite: 779]
    \setlength{\parindent}{1.25cm}

    Abstrak ini berisi ringkasan tentang isi proposal, meliputi ringkasan pendahuluan, tujuan, manfaat, metode penelitian, dan kata kunci. Jumlah kata maksimal 350 kata[cite: 780]. Paragraf pada abstrak ini dimulai masuk 1 tab dari batas margin kiri.

    \vspace{2em}
    \noindent \textbf{Kata kunci:} Latex, Linux, Zed, Skripsi (Maksimal 5 kata) [cite: 781]
\end{singlespace}