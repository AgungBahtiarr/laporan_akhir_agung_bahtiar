\chapter{HASIL DAN PEMBAHASAN}

\section{Hasil Penelitian}
Uraikan data yang didapat. Sajikan data dalam bentuk tabel jika diperlukan. Ingat, judul tabel ada di \textbf{atas} tabel[cite: 434].

\begin{table}[h]
    \centering
    \caption{Hasil Pengujian Blackbox}
    \label{tab:hasil_uji}
    \begin{singlespace} % Spasi 1
    \begin{tabular}{|l|l|l|l|}
        \hline
        \textbf{No} & \textbf{Skenario} & \textbf{Hasil Harapan} & \textbf{Kesimpulan} \\ \hline
        1 & Login User & Masuk Dashboard & Berhasil \\ \hline
        2 & Input Data Salah & Muncul Error & Berhasil \\ \hline
    \end{tabular}
    \end{singlespace}
    \vspace{0.2cm}
    \small{Sumber: Data Olahan Peneliti (2026)} % Sumber di bawah tabel 
\end{table}

\section{Pembahasan}
Bagian ini adalah inti TA. Bahas hasil di atas dan bandingkan dengan teori di Bab 2 atau penelitian terdahulu[cite: 793]. Jangan hanya menampilkan angka, tapi jelaskan "mengapa" hasilnya demikian.